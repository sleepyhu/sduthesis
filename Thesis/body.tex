% !Mode:: "TeX:UTF-8"
% !TEX root = tjumain.tex

\iffalse
\bibliography{reference/reference.bib} % 欺骗latextools获取bib文件
\fi

%%%%%%% 正文 %%%%%%%
% \setcounter{chapter}{0}
\chapter{绪论}

\section{研究背景}
这里是研究背景。

这是引用\cite{simon2011metagenomic}示例。

\section{问题定义与研究现状}
这里是问题定义与研究现状
\section{论文主要内容与结构}
\noindent
论文主要研究内容为:

\begin{itemize}
    \item 阿巴阿巴阿巴
    \item 那叫一个地道
    \item 奥利给干了xdm
\end{itemize}

\noindent

\chapter{章标题}
\section{节标题}
表格示例

% 该代码块用于创建一个表格
\begin{table}[H]
% 表格标题和标签
\caption{k-mer序列}\label{tab:table1}
% 设置表格的垂直间距,居中对齐,调整字体大小
\vspace{0.5em}\centering\wuhao
\begin{tabular}{ccccc}
\toprule[1.5pt]
k & k-mers\\
\midrule[1pt]
3 & GCT,CTA,TAC,ACT,CTG,TGA,GAC,ACG     \\
4 & GCTA,CTAC,TACT,ACTG,CTGA,TGAC,GACG  \\
5 & GCTAC,CTACT,TACTG,ACTGA,CTGAC,TGACG \\
\bottomrule[1.5pt]
\end{tabular}
\vspace{\baselineskip}
\end{table}
图片示例
\begin{figure}[htb]
\centering
\includegraphics[width=0.75\textwidth]{figures/windy.jpeg}
\caption{温迪世界第一可爱}\label{fig:k-mer1}
\vspace{2em}
\end{figure}

销售商决策如式~\eqref{rcond}~所示:
\begin{equation}
    \label{rcond}
    \left\{\begin{array}{l}
        p_{1s}=v_h-(v_h-p_2)\mathbb{E}(\varphi)                            \\
        p_{2s}=v_l                                                         \\
        q_s \in \underset{q \geq 0}{\mathrm{argmax}} \beta_R (q, p_1, p_2) \\
    \end{array}\right.
\end{equation}
